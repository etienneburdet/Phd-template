% ********** Acknowledgements **********
\chapter*{Remerciements}

Ce mémoire n'est bien sûr pas l'unique résultat de mon travail derrière le clavier et la manipulation de quelques équations. Tout était prêt, bien avant même que je ne songe à m'engager dans cette thèse, pour que je n'ai plus qu'à confortablement à m'installer dans un sujet taillé sur mesure.

Avant de m'attaquer à ma problématique scientifique, j'ai pu plus pragmatiquement profiter d'un poste de travail, organiser toutes les réunions et séminaires que je souhaitais et voyager avec une certaine insouciance. Si tout ce confort matériel s'est présenté à moi avec une telle facilité (en dépit de la priorité assez basse que je suis capable de donner au travail administratif), c'est grâce au travail des équipes informatiques et administratives de l'EIVP. Tout le travail qu'ils m'ont consacré a été autant de temps et de tranquillité d'esprit gagnés de mon côté, pour que je puisse me concentrer sur la production scientifique. Cette thèse est donc aussi le résultat de leur travail.

Aux conditions matérielles, il faut bien sûr ajouter les conditions scientifiques. Depuis le début de la problématisation jusqu'à dernière relecture, Denis Morand et Morgane Colombert se seront assurés que j'étais dans le droit chemin, tout en me laissant suffisamment de liberté pour que je me sente à mon aise, une équation qui demande un tact quotidien quand on sait à quel point une certaine fierté sur ma production (c'est là mon moindre défaut) me rend parfois susceptible face à la critique. Leur travail a commencé très largement en amont, avec la mise en place de tout l'axe énergie et climat à l'EIVP et l'Université de Marnes-la-Vallée. J'ai eu notamment la chance, par le biais du projet SERVEAU, d'être au contact le plus direct des problématiques professionnelles. C'est aussi grâce au subtil agencement pluridisciplinaire de cet axe de recherche que j'ai pu être entouré de Martin Hendel et Charlotte Tardieu, qui auront pu très régulièrement me faire sortir de mon monde théorique, pour des confrontations à la réalité expérimentale et opérationnelle. Même si aucun des échanges que nous avons eu n’apparaît clairement dans ce mémoire, je tiens à souligner à quel point ils auront guidé implicitement nombre de mes choix. Derrière toute cette organisation, il y a bien sûr mon directeur Youssef Diab, qui m'accompagne depuis le master Développement Urbain Durable à l'Université de Marne-la-Vallée et a su me faire prendre le virage clé du génie urbain dans mon parcours scientifique. Depuis les cours et ateliers du master jusqu'à la recherche aujourd'hui, j'ai pu trouver tous les éléments nécessaires à ma production scientifique et rencontrer une incroyable diversité d'acteurs et penseurs de la ville. Au rang desquels je dois citer l’inénarrable Jean-Baptiste Vaquin, source infinie de bibliographie, de critiques aussi sarcastiques que pertinentes et d’anecdotes piquantes sur la ville de Paris. La voie du génie urbain m'aura surtout permis de rencontrer celui que je dois bien appeler mon maître à penser, Serge Salat. Depuis mon mémoire de master, je n'ai pu exploiter qu'une infime proportion des pistes de recherches que son intuition géniale a pu produire, toujours avec une rare fertilité. Sa pensée est un des piliers essentiels de cette thèse et elle guidera très certainement nombre de mes recherches futures. 

Je ne peux pas ici citer toutes les passionnantes rencontres scientifiques que j'ai eu la chance de faire au cours de ces trois ans. Je vais tout de même citer Jérôme Kämpf, non pas à titre scientifique comme c'est déjà le cas de nombreuses fois dans ce mémoire, mais à titre plus personnel, pour son chaleureux accueil au LESO à l'EPFL. Je dois aussi mentionner collectivement toutes les personnes que j'ai pu rencontrer à la conférence PLEA, et en particulier ceux qui nous auront fait le plaisir par la suite de nous rendre visite à l'EIVP, tous reconnaîtront leur contribution je l’espère.

Enfin ce serait un crime de ne pas mentionner ceux qui ont fait mon quotidien, bien au-delà du bureau, pendant ces trois ans. Tous les doctorants de l'EIVP auront contribué à me donner le sourire quand je daignais lever le nez pour oublier quelque instants mon code, mes équations ou mon thé. Je ne sais pas si nous auront beaucoup fait progresser la science, mais la condition gastronomique dans le monde doctoral aura fait un grand pas, et j'aurais du mal à cacher à quel point ces nourritures plus terrestres sont essentielles à mon équilibre et mon travail scientifique.

La recherche sur la ville durable est un milieu vibrant, passionnant, auquel j’espère que cette thèse me permettra d'appartenir pour de nombreuses années encore.

% ********** End of Acknowledgements **********
