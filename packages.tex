%******************************************************************
% This file defines the document design.
% Usually it is not necessary to edit this file, but you can change
% the design if you want.
% ******************************************************************
% *************** Load packages ***************
%% Font & Enconding options
\usepackage[utf8]{luainputenc}
\usepackage{lmodern}
\usepackage{fontspec}

%% Unquote for serif typeface
%\renewcommand*\familydefault{\sfdefault}

%% Unquote for microtype optimization. Recommended to activate only for final document
%% \usepackage{microtype}

%% Language options
\usepackage{polyglossia}
\setdefaultlanguage{french}


%% Graphics and symbols
\usepackage{color}
\definecolor{blue}          {cmyk}{1   , 1   , 0   , 0   }
\usepackage{amsmath}
\usepackage{graphicx}
\usepackage{booktabs}
\usepackage[round,authoryear]{natbib}
\usepackage{epigraph}

\usepackage[
		plainpages=false,
		pdfpagelabels,
		bookmarksnumbered,%
		bookmarksopen,
        colorlinks=true,%
        linkcolor=blue,%
        citecolor=blue,%
        filecolor=blue,%
        urlcolor=blue,%
%        pdftex,%
%        unicode,
        bookmarksopen
        ]{hyperref}  

\usepackage{rotating}
\usepackage{tabularx}
\usepackage{flafter}
%\usepackage{subcaption}

%% Glossaries. Recommended to use xindy via makeglossaries
\usepackage[nomain,acronym,toc,xindy={codepage=utf8}]{glossaries}
\makeglossaries
% *************** Load packages ***************
%% Font & Enconding options
\usepackage[utf8]{luainputenc}
\usepackage{lmodern}
\usepackage{fontspec}

%% Unquote for serif typeface
%\renewcommand*\familydefault{\sfdefault}

%% Unquote for microtype optimization. Recommended to activate only for final document
%% \usepackage{microtype}

%% Language options
\usepackage{polyglossia}
\setdefaultlanguage{french}


%% Graphics and symbols
\usepackage{color}
\definecolor{blue}          {cmyk}{1   , 1   , 0   , 0   }
\usepackage{amsmath}
\usepackage{graphicx}
\usepackage{booktabs}
\usepackage[round,authoryear]{natbib}
\usepackage{epigraph}

\usepackage[
		plainpages=false,
		pdfpagelabels,
		bookmarksnumbered,%
		bookmarksopen,
        colorlinks=true,%
        linkcolor=blue,%
        citecolor=blue,%
        filecolor=blue,%
        urlcolor=blue,%
%        pdftex,%
%        unicode,
        bookmarksopen
        ]{hyperref}  

\usepackage{rotating}
\usepackage{tabularx}
\usepackage{flafter}
%\usepackage{subcaption}

%% Glossaries. Recommended to use xindy via makeglossaries
%\usepackage[nomain,acronym,toc,xindy={codepage=utf8}]{glossaries}
%\makeglossaries

% *************** Part Style ***************
%%%% Note : sketchy as of now, feel free to submit something cleaner !
\setlength\epigraphwidth{0.3\paperwidth}
\epigraphfontsize{\normalsize}
\renewcommand{\textflush}{flushleftright}

\newlength{\partboxlength}
\setlength{\partboxlength}{\paperwidth}
\addtolength{\partboxlength}{-\linewidth}
\renewcommand*{\thepart}{\arabic{part}}
\renewcommand*{\parttitlefont}{\raggedleft\normalfont\Huge\MakeUppercase}
\renewcommand*{\beforepartskip}{\null\vskip 0pt plus 0.4fil}
\renewcommand*{\printparttitle}[1]{%
\begin{tabular}{b{\linewidth}b{\partboxlength}}
\parttitlefont{#1} & \hspace{1em} \rule{0.5em}{8em} \hspace{0.5em} \rule{8em}{8em}
\end{tabular}
}

\renewcommand{\part*}{3}{
\let\oldafterpartskip\afterpartskip % save definition
\renewcommand{\afterpartskip}{%
\vspace{1em}
\epigraph{#2}{#3}
\part*{#1}
\addcontentsline{toc}{part}{\emph{#1}}
}
}