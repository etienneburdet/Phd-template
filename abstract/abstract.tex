\begin{abstract}
Nous avons construit un modèle couplé de canopée, d'énergétique du bâtiment et de transferts radiatifs pour le besoin de la conception des formes urbaines. Ce besoin a émergé d'une volonté des bureaux d'études de pouvoir calculer toutes les consommations énergétiques à l'échelle d'un projet d'aménagement, croisé avec les connaissances existantes sur la morphologie urbaine et les flux énergétiques (ensoleillements, consommations, ilot de chaleur urbain). Cela nous a amené à la problématique de l'énergie dans la conception des formes urbaines, i.e. le tracé des plans, pour laquelle aucun modèle d'ingénierie n'existe, malgré de nombreuses connaissances en physique urbaine. Pour cela nous avons reconstruit un modèle de conception à partir des bases physiques existantes. Le premier objectif de cette reformulation est que tous les modèles partagent la même description de la morphologie, une définition étage par étage de formes 3D quelconques, correspondant aux plans utilisés actuellement dans l'urbanisme. Le second objectif est que les paramètres utilisés dans les équations soient ceux sur lesquels des choix de conception sont faits aux étapes de l'aménagement urbain, sans faire d'hypothèses sur les données non disponibles à ces étapes. Cela revient à modéliser un plan, conçu des années avant toute réalisation, et non la ville réelle, ce qui change considérablement l'objet modélisé. Nous avons dans un premier temps adapté un modèle de canopée multi-couches afin qu'il puisse prendre en compte toutes les variations de la morphologie étage par étage. Ensuite nous avons reformulé un modèle monozone de bâtiment, principalement pour que ses paramètres correspondent au niveau de définition des projets d'aménagement. Enfin nous avons reformulé un modèle de radiosité, en particulier pour que ses représentations correspondent bien aux plans de conceptions. À l'aide de ces modèles couplés nous avons étudié l'importance de la morphologie tridimensionelle, par le biais de la convexité, du nombre de bâtiments et du gradient vertical de densité. Ces effets s'expriment particulièrement dans le cadre des rétroactions spécifiques à notre échelle, entre les flux solaires, les flux turbulents et les échanges entre la canopée et les bâtiments. Au final le modèle que nous avons construit diffère surtout des modèles initiaux par son interprétation et son utilisation, la physique étant la même que celle des modèles de départ. Le but de ce modèle est de stimuler l'intuition et de tester la logique du modélisateur, dans des analyses de solution au cas par cas. Nous avons enfin étudié les perspectives qu'ouvrent un tel modèle, aussi bien dans les études théoriques du rapport entre forme urbaine et énergie, que dans son application à l'ingénierie. À notre sens, le retour observé du génie urbain dans la conception des formes urbaines dans lequel nous nous inscrivons, donne une perspective d'analyse originale, celle de la ville comme un objet conçu, vue par les choix de ceux qui la construisent.
\end{abstract}